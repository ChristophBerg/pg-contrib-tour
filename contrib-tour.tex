\documentclass[utf8,hyperref={pdftex,colorlinks,linkcolor=black,citecolor=black,urlcolor=black,filecolor=black,plainpages=false},xcolor=table,hyperref]{beamer}

% ---- Titles, etc ----
\newcommand*{\Title}{Eine Tour durch PostgreSQL-Contrib}
\newcommand*{\SubTitle}{Extend Your Database Server}
\newcommand*{\Author}{Christoph Berg <christoph.berg@credativ.de>}

% ---- Preamble ----
% ---- these are sensitive to order, load first ----
\usepackage{xspace}			% prevent space eating after commands
\xspaceaddexceptions{\xspace}           % allow nested use of commands using \xspace
\usepackage{ragged2e}			% Allow hyphenation in ragged text (e.g. by using uppercase \Centering)

% ---- Fonts, encodings, symbols ----
\usepackage[T1]{fontenc}		% font encoding defaults to latin
\usepackage{mathptmx}			% nice fonts
\usepackage{lmodern}			% modern latin fonts
\usepackage{eurosym}			% Eurosymbol
\usepackage{textcomp}			% provides copyleft symbol
\usepackage{underscore}			% remove necessity to escape underscores

% ---- Language ----
\usepackage[english,british,ngerman]{babel}	% support for localization

% define short commands providing conditionals for different languages
\newcommand{\ifEN}[1]{%
  \iflanguage{english}{#1}{%
    \iflanguage{british}{#1}{}%
  }%
}
\newcommand{\ifDE}[1]{%
  \iflanguage{german}{#1}{%
    \iflanguage{ngerman}{#1}{}%
  }%
}

% ---- Text ----
\usepackage[
	babel,
	german=quotes]{csquotes}	% Vereinheitlichte Anführungsstriche
\usepackage{upquote}			% Anführungsstriche in Verbatim-Umgebungen
\usepackage{comment}			% comment environment for easy multi-line comments
\usepackage{fancyvrb}			% sophisticated verbatim text
\usepackage{listings}			% advanced code listings

% ---- watermarks ----
\usepackage{draftwatermark}             % add draft watermark to pages
\usepackage{everypage}                  % needed to include watermark on each page

% ---- Layout - Floats, Tables, etc ----
\usepackage{floatrow}			% grouped placement of floats (without the subfig bugs)
\usepackage{tabularx}			% automatically calculates column-width
\usepackage{multirow}			% join table rows
\usepackage{booktabs}			% nicer lines between table-cells with better spacing
\usepackage{fancyhdr}			% nicer header

\usepackage[shadow]{todonotes}		% graphical notes shown in the margin
\newcommand{\FIXME}[1]{%
  \todo[caption={FIXME: #1},backgroundcolor=red!50,linecolor=red!50]{#1}%
  {}%
}
\newcommand{\NOTE}[1]{%
  \todo[caption={NOTE: #1},backgroundcolor=blue!50,linecolor=blue!50]{#1}%
  {}%
}
\newcommand{\CHECK}[1]{%
  \todo[caption={NOTE: #1},backgroundcolor=yellow!50,linecolor=yellow!50]{#1}%
  {}%
}
\newcommand{\DONE}[1]{%
  \todo[caption={NOTE: #1},backgroundcolor=green!50,linecolor=green!50]{#1}%
  {}%
}

\input{preamble/hyphenation}
% ---- PDF settings ----
\hypersetup{%
        pdftitle={\Title},
        pdfauthor={\Author},
}


% ---- Document settings ----
\title{\Title}

\subtitle{\SubTitle}

\author{\Author}
\institute{credativ GmbH, Mönchengladbach}

\date{\today} 
\subject{Talks}


\usetheme{credativ}

% ---- document ----
\begin{document}

\input{mainmatter/TitlePage}        % titlepage with neat logo
%\begin{frame}{Übersicht}
  \tableofcontents
\end{frame}
  % table of contents

% begin of the actual content
% use one \input for every section
%\section{Einleitung}

\begin{frame}{Entwicklung}

        \begin{itemize}
        \item Punkt 1
        \item Punkt 2
                \begin{itemize}
                \item Unterpunkt 1
                \item Unterpunkt 2
                \end{itemize}
        \item Punkt 3
        \item Punkt 4
        \end{itemize}

\end{frame}


%\section{Details}

% Here each item is unveiled on a new frame.
%Beware: item 4 is only shown once on frame 4.
\begin{frame}{Kontext}
        \frametitle{Example Unveil - Translucent}
		\framesubtitle{Unveil frame contents step by step on multiple slides}
        \setbeamercovered{transparent=60}
        \begin{itemize}
        \item Punkt 1
        \item<2-> Punkt 2
        \item<3-3> Punkt 3
        \item<4-> Punkt 4
        \end{itemize}
\end{frame}

\begin{frame}{Kontext}
        \frametitle{Example Unveil - Invisible}
		\framesubtitle{Unveil frame contents step by step on multiple slides}
        \setbeamercovered{invisible}
        \begin{itemize}
        \item Punkt 1
        \item<2-> Punkt 2
        \item<3-3> Punkt 3
        \item<4-> Punkt 4
        \end{itemize}
\end{frame}

%\begin{frame}{Boxes}
        \begin{block}{Blocktitel}
                Blocktext
        \end{block}
        \begin{exampleblock}{Beispielblocktitel}
                Beispielblocktext
        \end{exampleblock}
        \begin{alertblock}{Warnungsblocktitel}
                Warnungsblocktext
        \end{alertblock}
\end{frame}


\begin{frame}[containsverbatim]
	\frametitle{CREATE EXTENSION}
	\verb|postgres=# CREATE EXTENSION <tab><tab>|

fuzzystrmatch pg_trgm plsh cube autoinc chkpass lo btree_gin btree_gist
test_parser plpython3u adminpack postgis_topology prefix
tablefunc uuid-ossp plperl earthdistance unaccent plpythonu postgres_fdw hstore
pgstattuple insert_username pgrowlocks tcn isn pg_stat_statements
pg_buffercache xml2 postgis moddatetime plpgsql pgcrypto pltclu pltcl intarray
refint citext pg_freespacemap tsearch2 ltree file_fdw seg sslinfo timetravel
dict_int dblink plperlu pgmp plpython2u intagg pageinspect dict_xsyn
\end{frame}


\begin{frame}
	\frametitle{CREATE EXTENSION}
	\begin{itemize}
		\item Neue Funktionen
		\item Datentypen
		\item Index-Methoden
		\item Procedural Languages (PL)
		\item Trigger
		\item Statistiken
	\end{itemize}
\end{frame}


\begin{frame}
	\frametitle{CREATE EXTENSION}
	\begin{itemize}
		\item PostgreSQL "contrib"
		\item Viele weitere im Internet (github, pgfoundry, pgxn, \dots)
		\item Debian/Ubuntu: apt.postgresql.org
		\item CREATE EXTENSION seit PostgreSQL 9.1
		\item Vorher $\backslash$i /usr/share/postgresql/8.4/contrib/foobar.sql
		\item Vereinfachtes Handling, Backup
	\end{itemize}
\end{frame}


\begin{frame}[containsverbatim]
	\frametitle{CREATE EXTENSION "{}uuid-ossp";}
	\begin{verbatim}
# select uuid_generate_v1();
           uuid_generate_v1           
--------------------------------------
 6eb92d28-4634-11e3-ae7a-001f29960aed
 \end{verbatim}
\end{frame}

\begin{frame}[containsverbatim]
	\frametitle{CREATE EXTENSION pg_trgm;}
	\begin{itemize}
		\item String wird in Trigramme zerlegt \\
			cat $\rightarrow$ "\ \ c", "\ ca", "cat", "{}at\ "
		\item Abstand zwischen Strings: \verb|<->|
			\begin{verbatim}
			# select 'Oberhausen' <-> 'Essen';
			 ?column? 
			 ----------
			  0.866667
			  \end{verbatim}
		  \item CREATE INDEX ON test_trgm USING gist (t gist_trgm_ops); \\
			CREATE INDEX ON test_trgm USING gin (t gin_trgm_ops);
		\item "Meinten Sie \dots?"\begin{verbatim}
			SELECT t, t <-> 'word' AS dist
			  FROM suchbegriffe
			    ORDER BY dist LIMIT 10;
			    \end{verbatim}
		    
	\end{itemize}
\end{frame}

\begin{frame}[containsverbatim]
	\frametitle{CREATE EXTENSION plsh;}
	\begin{verbatim}
CREATE FUNCTION ls(text) RETURNS text AS '
#!/bin/sh
ls -l "$1"
' LANGUAGE plsh;

SELECT ls ('/etc/passwd');
                          ls                          
------------------------------------------------------
 -rw-r--r-- 1 root root 2818 Feb  7  2013 /etc/passwd
\end{verbatim}
\end{frame}

\begin{frame}[containsverbatim]
	\frametitle{CREATE EXTENSION cube;}
	\begin{itemize}
		\item Multidimensional Cubes
			\begin{verbatim}
# select '(5,7)'::cube <@ '[(3,3),(8,8)]' AS punkt_im_rechteck;
 punkt_im_rechteck 
-------------------
 t
 \end{verbatim}
 \item GiST-Support
 \end{itemize}
\end{frame}

\begin{frame}[containsverbatim]
	\frametitle{CREATE EXTENSION earthdistance;}
	\begin{verbatim}
select earth_distance(ll_to_earth(51,6), ll_to_earth(52,10));
  earth_distance  
------------------
 298660.368346433
 \end{verbatim}
\end{frame}

\begin{frame}[containsverbatim]
	\frametitle{CREATE EXTENSION postgis;}
	\begin{itemize}
		\item Größte Extension, eigenes Open Source-Projekt
			\includegraphics[height=4cm]{postgis.png}
		\item Geo-Datentypen, OpenGIS, ISO SQL/MM
		\item "Welche Grundstücke liegen an dieser Straße?"
		\item Raster-Daten
		\item Routenplanung
	\end{itemize}
\end{frame}

\begin{frame}[containsverbatim]
	\frametitle{CREATE EXTENSION pg_stat_statements;}
	\begin{itemize}
		\item Liste der ausgeführten Queries
		\item Verbessert seit PostgreSQL 9.2
		\item \scriptsize \begin{verbatim}
# SELECT query, calls, total_time, rows, 100.0 * shared_blks_hit /
               nullif(shared_blks_hit + shared_blks_read, 0) AS hit_percent
          FROM pg_stat_statements ORDER BY total_time DESC LIMIT 5;
-[ RECORD 1 ]---------------------------------------------------------------------
query       | UPDATE pgbench_branches SET bbalance = bbalance + ? WHERE bid = ?;
calls       | 3000
total_time  | 9609.00100000002
rows        | 2836
hit_percent | 99.9778970000200936
-[ RECORD 2 ]---------------------------------------------------------------------
query       | UPDATE pgbench_tellers SET tbalance = tbalance + ? WHERE tid = ?;
calls       | 3000
total_time  | 8015.156
rows        | 2990
hit_percent | 99.9731126579631345
-[ RECORD 3 ]---------------------------------------------------------------------
query       | copy pgbench_accounts from stdin
calls       | 1
total_time  | 310.624
rows        | 100000
hit_percent | 0.30395136778115501520
\end{verbatim}
\end{itemize}
\end{frame}

\begin{frame}[containsverbatim]
	\frametitle{CREATE EXTENSION plpythonu;}
	\begin{itemize}
		\item plpython2u, plpython3u
		\item "{}untrusted": CREATE FUNCTION nur für Superuser
		\item
			\begin{verbatim}
CREATE FUNCTION pymax (a integer, b integer)
  RETURNS integer
AS $$
  if a > b:
    return a
  return b
$$ LANGUAGE plpythonu;
\end{verbatim}
	\end{itemize}
\end{frame}

\begin{frame}[containsverbatim]
	\frametitle{CREATE EXTENSION adminpack;}
	\begin{itemize}
		\item Support-Funktionen für pgAdmin3 ("{}Server-Instrumentierung")
		\item Files lesen/schreiben
		\item postgresql.conf, pg_hba.conf, Logfiles
	\end{itemize}
\end{frame}

%\begin{frame}[containsverbatim]
%	\frametitle{CREATE EXTENSION postgis_topology;}
%\end{frame}

\begin{frame}[containsverbatim]
	\frametitle{CREATE EXTENSION prefix;}
	\begin{itemize}
		\item Präfix-Matching mit Index-Support
		\item 
			\begin{verbatim}
    SELECT * 
      FROM prefixes
     WHERE prefix @> '01805'
  ORDER BY length(prefix) DESC
     LIMIT 1;
     \end{verbatim}
     \end{itemize}
\end{frame}

\begin{frame}[containsverbatim]
	\frametitle{CREATE EXTENSION tablefunc;}
	\begin{itemize}
		\item normal_rand, connectby
		\item crosstab
			{\scriptsize
			\begin{verbatim}
CREATE TABLE ct(id SERIAL, rowid TEXT, attribute TEXT, value TEXT);
INSERT INTO ct(rowid, attribute, value) VALUES('test1','att1','val1');
INSERT INTO ct(rowid, attribute, value) VALUES('test1','att2','val2');
INSERT INTO ct(rowid, attribute, value) VALUES('test1','att3','val3');
INSERT INTO ct(rowid, attribute, value) VALUES('test1','att4','val4');
INSERT INTO ct(rowid, attribute, value) VALUES('test2','att1','val5');
INSERT INTO ct(rowid, attribute, value) VALUES('test2','att2','val6');
INSERT INTO ct(rowid, attribute, value) VALUES('test2','att3','val7');
INSERT INTO ct(rowid, attribute, value) VALUES('test2','att4','val8');

SELECT *
FROM crosstab(
  'select rowid, attribute, value
   from ct
   order by 1,2')
AS ct(row_name text, a1 text, a2 text, a3 text, a4 text);
 row_name |  a1  |  a2  |  a3  |  a4  
----------+------+------+------+------
 test1    | val1 | val2 | val3 | val4
 test2    | val5 | val6 | val7 | val8
 \end{verbatim}}
	\end{itemize}
\end{frame}

\begin{frame}[containsverbatim]
	\frametitle{CREATE EXTENSION plperl;}
	\begin{itemize}
		\item Trusted, CREATE FUNCTION für alle erlaubt
		\item Keine Perl-Module erlaubt (use)
		\item \begin{verbatim}
create function match(text, text)
  returns boolean as $$
    ($a, $b) = @_;
    return $a =~ $b;
  $$ language plperl;
\end{verbatim}
\end{itemize}
\end{frame}

\begin{frame}[containsverbatim]
	\frametitle{CREATE EXTENSION postgres_fdw;}
	\begin{verbatim}
create table t (t text);
insert into t values ('Hallo Welt');

create server loopback foreign data wrapper postgres_fdw
  options (port '5432');
create user mapping for current_user server loopback ;
create foreign table t2 (t text) server loopback
  options (table_name 't');

select * from t2;
     t      
------------
 Hallo Welt
\end{verbatim}
\end{frame}

\begin{frame}[containsverbatim]
	\frametitle{CREATE EXTENSION dblink;}
	\begin{itemize}
		\item Zugriff auf andere PostgreSQL-Datenbanken
		\item Älteres Interface als postgres_fdw
		\item \begin{verbatim}
CREATE VIEW myremote_pg_proc AS
  SELECT *
    FROM dblink('dbname=postgres', 'select proname, prosrc from pg_proc')
    AS t1(proname name, prosrc text);

SELECT * FROM myremote_pg_proc WHERE proname LIKE 'bytea%';
\end{verbatim}
\end{itemize}
\end{frame}

\begin{frame}[containsverbatim]
	\frametitle{CREATE EXTENSION hstore;}
	\begin{itemize}
		\item Perl-Style Hashes
		\item Semi-strukturierte Daten
		\item \begin{verbatim}
SELECT 'name => "Berg", vorname => "Christoph",
  firma => "credativ"'::hstore;
                           hstore                            
-------------------------------------------------------------
 "name"=>"Berg", "firma"=>"credativ", "vorname"=>"Christoph"
 \end{verbatim}
 \item \verb|SELECT h->'name';|
 \item Index-Support
 \end{itemize}
\end{frame}

\begin{frame}[containsverbatim]
	\frametitle{CREATE EXTENSION plpgsql;}
	\begin{itemize}
		\item PL/pgSQL: SQL + prozedurale Elemente
		\item Per Default installiert
		\item \small
			\begin{verbatim}
CREATE OR REPLACE FUNCTION get_all_foo() RETURNS SETOF foo AS
$BODY$
DECLARE
    r foo%rowtype;
BEGIN
    FOR r IN
        SELECT * FROM foo WHERE fooid > 0
    LOOP
        -- can do some processing here
        RETURN NEXT r; -- return current row of SELECT
    END LOOP;
    RETURN;
END
$BODY$
LANGUAGE plpgsql;

SELECT * FROM get_all_foo();
\end{verbatim}
	\end{itemize}
\end{frame}

\begin{frame}[containsverbatim]
	\frametitle{CREATE EXTENSION autoinc;}
	\begin{itemize}
		\item 
	\begin{verbatim}
# CREATE SEQUENCE next_id;
# CREATE TABLE ids (id int4, idesc text);
# CREATE TRIGGER ids_nextid
    BEFORE INSERT OR UPDATE ON ids
    FOR EACH ROW
    EXECUTE PROCEDURE autoinc (id, next_id);
# INSERT INTO ids VALUES (0, 'first');
# INSERT INTO ids VALUES (null, 'second');
# INSERT INTO ids(idesc) VALUES ('third');
# SELECT * FROM ids ;
 id | idesc  
----+--------
  1 | first
  2 | second
  3 | third
\end{verbatim}
\item Ähnlich "{}serial"
	\end{itemize}
\end{frame}

\begin{frame}[containsverbatim]
	\frametitle{CREATE EXTENSION chkpass;}
	\begin{itemize}
		\item Speichert crypt()-Passwörter
		\item Eingabe im Klartext
		\item Ausgabe verschlüsselt als Hash
		\item Vergleicht als Hash
		\item 
			\begin{verbatim}
test=# create table test (p chkpass);
test=# insert into test values ('hello');
test=# select * from test where p = 'hello';
       p        
----------------
 :fN6iG6x8Mv22M
 \end{verbatim}
 \end{itemize}
\end{frame}

\begin{frame}[containsverbatim]
	\frametitle{CREATE EXTENSION fuzzystrmatch;}
	\begin{itemize}
		\item Soundex: Ähnliche Namen, US-Zensus 1880
			{\footnotesize
		\begin{verbatim}
# CREATE TABLE s (nm text);
# INSERT INTO s VALUES ('john'), ('joan'), ('wobbly'), ('jack');
# SELECT * FROM s WHERE difference(s.nm, 'john') > 2;
  nm  
------
 john
 joan
 jack
			  \end{verbatim}}
		  \item Levenshtein: Abstand in Zahl der Änderungen am String
			  {\footnotesize
			  \begin{verbatim}
# SELECT levenshtein('GUMBO', 'GAMBOL');
 levenshtein
-------------
           2
	   \end{verbatim}}
   \item metaphone: Ähnlich Soundex
	  \end{itemize}

\end{frame}

\begin{frame}[containsverbatim]
	\frametitle{CREATE EXTENSION btree_gin;}
	\begin{itemize}
		\item CREATE TABLE test (a int4);
		\item CREATE INDEX testidx ON test USING gin (a);
		\item Nützlich für mehrspaltige GIN-Indexe
	\end{itemize}
\end{frame}

\begin{frame}[containsverbatim]
	\frametitle{CREATE EXTENSION btree_gist;}
	\begin{itemize}
		\item Analog btree_gin
		\item Support für \verb|<->| (Abstand)
		\item Support für Exclusion Constraints mit \verb|<>|
			\begin{verbatim}
=> CREATE TABLE zoo (
  cage   INTEGER,
  animal TEXT,
  EXCLUDE USING gist (cage WITH =, animal WITH <>)
);
=> INSERT INTO zoo VALUES(123, 'zebra');
=> INSERT INTO zoo VALUES(123, 'zebra');
=> INSERT INTO zoo VALUES(123, 'lion');
ERROR:  conflicting key value violates exclusion
  constraint "zoo_cage_animal_excl"
DETAIL:  Key (cage, animal)=(123, lion) conflicts with
  existing key (cage, animal)=(123, zebra).
=> INSERT INTO zoo VALUES(124, 'lion');
\end{verbatim}
\end{itemize}
\end{frame}

\begin{frame}[containsverbatim]
	\frametitle{CREATE EXTENSION pgstattuple;}
	\begin{itemize}
		\item Liest ganze Tabelle oder Index
		\item \begin{verbatim}
SELECT * FROM pgstattuple('pg_catalog.pg_proc');
-[ RECORD 1 ]------+-------
table_len          | 458752
tuple_count        | 1470
tuple_len          | 438896
tuple_percent      | 95.67
dead_tuple_count   | 11
dead_tuple_len     | 3157
dead_tuple_percent | 0.69
free_space         | 8932
free_percent       | 1.95
\end{verbatim}
\item \begin{verbatim}
SELECT * FROM pgstatindex('pg_cast_oid_index');
\end{verbatim}
\end{itemize}
\end{frame}

\begin{frame}[containsverbatim]
	\frametitle{CREATE EXTENSION unaccent;}
	\begin{itemize}
		\item 
			\begin{verbatim}
# SELECT unaccent('Hôtel');
 unaccent 
----------
 Hotel
 \end{verbatim}
 \item Integration mit tsearch
	 \end{itemize} 
\end{frame}

\begin{frame}[containsverbatim]
	\frametitle{CREATE EXTENSION lo;}
	\begin{itemize}
		\item Mangement von BLOB-Referenzen über Trigger
		\item 
			\begin{verbatim}
CREATE TABLE image (title text, raster lo);

CREATE TRIGGER t_raster BEFORE UPDATE OR DELETE ON image
    FOR EACH ROW EXECUTE PROCEDURE lo_manage(raster);
    \end{verbatim}
		\item Nützlich für JDBC- und ODBC-Anwendungen
	\end{itemize}
\end{frame}

\begin{frame}[containsverbatim]
	\frametitle{CREATE EXTENSION insert_username;}
	\begin{verbatim}
CREATE TABLE username_test (
   name	     text,
   username  text not null
);

CREATE TRIGGER insert_usernames
   BEFORE INSERT OR UPDATE ON username_test
   FOR EACH ROW
   EXECUTE PROCEDURE insert_username (username);

INSERT INTO username_test VALUES ('foobar');

SELECT * FROM username_test;
  name  | username 
--------+----------
 foobar | cbe
\end{verbatim}

\end{frame}

\begin{frame}[containsverbatim]
	\frametitle{CREATE EXTENSION isn;}
	\begin{verbatim}
# SELECT ean13('4220356483487');
      ean13      
-----------------
 422-035648348-7
# SELECT isbn('978-0-393-04002-9');
     isbn      
---------------
 0-393-04002-X
# SELECT isbn13('0901690546');
      isbn13       
-------------------
 978-0-901690-54-8
# SELECT issn('1436-4522');
   issn    
-----------
 1436-4522
 \end{verbatim}
\end{frame}

\begin{frame}[containsverbatim]
	\frametitle{CREATE EXTENSION citext;}
	\begin{itemize}
		\item Case-insensitive text type
		\item Transparente Konvertierung groß/klein
		\item \begin{verbatim}
CREATE TABLE users (
    nick CITEXT PRIMARY KEY,
    pass TEXT   NOT NULL
);

INSERT INTO users VALUES ('larry', md5(random()::text));

SELECT * FROM users WHERE nick = 'Larry';
\end{verbatim}
\end{itemize}
\end{frame}

\begin{frame}[containsverbatim]
	\frametitle{CREATE EXTENSION pg_buffercache;}
	\small
	\begin{verbatim}
# SELECT c.relname, count(*) AS buffers
   FROM pg_buffercache b INNER JOIN pg_class c
   ON b.relfilenode = pg_relation_filenode(c.oid) AND
      b.reldatabase IN (0, (SELECT oid FROM pg_database
                            WHERE datname = current_database()))
   GROUP BY c.relname
   ORDER BY 2 DESC LIMIT 10;
            relname             | buffers 
--------------------------------+---------
 ranges                         |     100
 prefixes                       |      91
 pg_depend                      |      84
 pg_proc                        |      80
 idx_prefix                     |      72
 pg_depend_reference_index      |      49
 pg_attribute                   |      49
 pg_depend_depender_index       |      37
 pg_proc_proname_args_nsp_index |      35
 prefixes_pkey                  |      35
 \end{verbatim}
\end{frame}

%\begin{frame}[containsverbatim]
%	\frametitle{CREATE EXTENSION xml2;}
%	\begin{itemize}
%		\item XML-Support ist im Core, xml2 nicht mehr benutzen
%	\end{itemize}
%\end{frame}

\begin{frame}[containsverbatim]
	\frametitle{CREATE EXTENSION pgcrypto;}
	\begin{itemize}
		\item Kryptographische Funktionen
		\item Hashes: md5, sha1, sha256, \dots (OpenSSL)
		\item hmac()
		\item crypt()
		\item OpenPGP
			\begin{itemize}
				\item pgp_sym_encrypt(), pgp_sym_decrypt()
				\item pgp_pub_encrypt(), pgp_pub_decrypt()
			\end{itemize}
		\item gen_random_bytes()
	\end{itemize}
\end{frame}

\begin{frame}[containsverbatim]
	\frametitle{CREATE EXTENSION pgmp;}
	\begin{itemize}
		\item GNU Multi Precision library
		\item Große Integer, Primzahlen
		\item Rationale Zahlen (Brüche)
	\begin{verbatim}
select '3/8'::mpq + '5/6'::mpq as fraction;
 fraction 
----------
 29/24
 \end{verbatim}
 \end{itemize}
\end{frame}

\begin{frame}[containsverbatim]
	\frametitle{CREATE EXTENSION tcn;}
	\begin{itemize}
		\item Triggered Change Notification
		\item LISTEN/NOTIFY
		\item \begin{verbatim}
create trigger tcndata_tcn_trigger
   after insert or update or delete on tcndata
   for each row
   execute procedure triggered_change_notification();
listen tcn;

Asynchronous notification "tcn" with payload
   ""tcndata",I,"a"='2',"b"='2012-12-23'"
   received from server process with PID 22770.
\end{verbatim}
\end{itemize}
\end{frame}

\begin{frame}[containsverbatim]
	\frametitle{CREATE EXTENSION pgrowlocks;}
	\small
	\begin{verbatim}
# select * from pgrowlocks('tbl');
 locked_row | locker | multi |  xids   |     modes      | pids 
------------+--------+-------+---------+----------------+------
 (0,1)      |  18227 | f     | {18227} | {"For Update"} | {0}
 \end{verbatim}
\end{frame}

\begin{frame}[containsverbatim]
	\frametitle{CREATE EXTENSION pg_freespacemap;}
	\begin{itemize}
		\item Freier Platz in Tabellenblöcken
		\item
			\begin{verbatim}
# SELECT * FROM pg_freespace('foo');
 blkno | avail 
-------+-------
     0 |     0
     1 |     0
     2 |     0
     3 |    32
     4 |   704
     5 |   704
     6 |   704
     7 |  1216
     \end{verbatim}
     \item Selten notwendig
	     \end{itemize}
\end{frame}

%\begin{frame}[containsverbatim]
%	\frametitle{CREATE EXTENSION tsearch2;}
%\end{frame}

\begin{frame}[containsverbatim]
	\frametitle{CREATE EXTENSION ltree;}
	This module implements a data type ltree for representing labels of data stored in a hierarchical tree-like structure. Extensive facilities for searching through label trees are provided.

	\begin{verbatim}
ltreetest=> SELECT path FROM test WHERE path <@ 'Top.Science';
                path
------------------------------------
 Top.Science
 Top.Science.Astronomy
 Top.Science.Astronomy.Astrophysics
 Top.Science.Astronomy.Cosmology
 \end{verbatim}
\end{frame}

\begin{frame}[containsverbatim]
	\frametitle{CREATE EXTENSION pageinspect;}
	\begin{itemize}
		\item Block-level debugging
		\item get_raw_page(), page_header(), heap_page_items()
		\item bt_metap(), bt_page_stats(), bt_page_items()
		\item fsm_page_contents
	\end{itemize}
\end{frame}

\begin{frame}[containsverbatim]
	\frametitle{CREATE EXTENSION file_fdw;}
	The file_fdw module provides the foreign-data wrapper file_fdw, which can be used to access data files in the server's file system. Data files must be in a format that can be read by COPY FROM; see COPY for details. Access to such data files is currently read-only. 
\end{frame}

\begin{frame}[containsverbatim]
	\frametitle{CREATE EXTENSION pltcl;}
	\begin{itemize}
		\item pltcl, pltclu
	\end{itemize}
\end{frame}

\begin{frame}[containsverbatim]
	\frametitle{CREATE EXTENSION seg;}
	This module implements a data type seg for representing line segments, or floating point intervals. seg can represent uncertainty in the interval endpoints, making it especially useful for representing laboratory measurements. 
\end{frame}

\begin{frame}[containsverbatim]
	\frametitle{CREATE EXTENSION sslinfo;}
	The sslinfo module provides information about the SSL certificate that the current client provided when connecting to PostgreSQL. The module is useless (most functions will return NULL) if the current connection does not use SSL. 
\end{frame}

\begin{frame}[containsverbatim]
	\frametitle{CREATE EXTENSION timetravel;}
	 Long ago, PostgreSQL had a built-in time travel feature that kept the insert and delete times for each tuple. This can be emulated using these functions. To use these functions, you must add to a table two columns of abstime type to store the date when a tuple was inserted (start_date) and changed/deleted (stop_date):

	 \begin{verbatim}
CREATE TABLE mytab (
        ...             ...
        start_date      abstime,
        stop_date       abstime
        ...             ...
);
\end{verbatim}
\end{frame}

\begin{frame}[containsverbatim]
	\frametitle{CREATE EXTENSION dict_int;}
	dict_int is an example of an add-on dictionary template for full-text search. The motivation for this example dictionary is to control the indexing of integers (signed and unsigned), allowing such numbers to be indexed while preventing excessive growth in the number of unique words, which greatly affects the performance of searching. 
\end{frame}

\begin{frame}[containsverbatim]
	\frametitle{CREATE EXTENSION intarray;}
	The intarray module provides a number of useful functions and operators for manipulating null-free arrays of integers. There is also support for indexed searches using some of the operators. 
\end{frame}

\begin{frame}[containsverbatim]
	\frametitle{CREATE EXTENSION intagg;}
	The intagg module provides an integer aggregator and an enumerator. intagg is now obsolete, because there are built-in functions that provide a superset of its capabilities. However, the module is still provided as a compatibility wrapper around the built-in functions. 
\end{frame}

\begin{frame}[containsverbatim]
	\frametitle{CREATE EXTENSION dict_xsyn;}
	dict_xsyn (Extended Synonym Dictionary) is an example of an add-on dictionary template for full-text search. This dictionary type replaces words with groups of their synonyms, and so makes it possible to search for a word using any of its synonyms. 
\end{frame}

\begin{frame}[containsverbatim]
	\frametitle{CREATE EXTENSION moddatetime;}
	moddatetime() is a trigger that stores the current time into a timestamp field. This can be useful for tracking the last modification time of a particular row within a table. 
\end{frame}

\begin{frame}[containsverbatim]
	\frametitle{CREATE EXTENSION refint;}
	Functions for Implementing Referential Integrity
\end{frame}

\begin{frame}[containsverbatim]
	\frametitle{CREATE EXTENSION test_parser;}
	\begin{itemize}
		\item test_parser is an example of a custom parser for full-text search. It doesn't do anything especially useful, but can serve as a starting point for developing your own parser. 
	\end{itemize} 
\end{frame}

\end{document}
